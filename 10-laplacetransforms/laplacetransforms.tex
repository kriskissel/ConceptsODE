\documentclass[12pt,letterpaper,twoside]{amsart}
\usepackage[latin1]{inputenc}
\usepackage{amsmath}
\usepackage{amsfonts}
\usepackage{graphicx}
\usepackage{amssymb}
\usepackage{multicol}
\usepackage{ulem}
\newcounter{example}
\newcounter{exercise}
\newcounter{problem}
\newcommand{\example}{\bigskip \noindent {\large {\sc Example \arabic{example}:}} \addtocounter{example}{1}}
\newcommand{\exercise}{\bigskip \noindent {\large {\sc Exercise \arabic{exercise}:}} \addtocounter{exercise}{1}}
\newcommand{\problem}{\bigskip \noindent {\large {\sc Problem \arabic{problem}:}} \addtocounter{problem}{1}}
\newcommand{\tech}{\marginpar{\vskip 10mm \begin{center}\includegraphics[width=0.25in]{calculatorimagesmall.eps} \end{center}}}
\newcommand{\solution}{\medskip \noindent {\bf Solution: }}
\newcommand{\R}{\mathbb{R}}





\begin{document}

\sffamily

%%%%  switch the commenting on this line and the next \chapter{Introduction}
\begin{center} {\LARGE Laplace Transforms} \end{center}

\setcounter{example}{1}
\setcounter{exercise}{1}

In this chapter we will introduce the idea of a transform method.  The basic idea is this: we begin with an inial value problem for a differential equation, and we transform this equation into an algebraic equation; once we solve for the unknown in the algebraic equation, we then transform back to find a corresponding solution of the IVP.

The tool we will use for this is the {\bf Laplace Transform} of a function, defined by
\[ L[f]=\int_0^\infty f(t)e^{-st} \ dt.\]
Here, $f$ is a function defined on $[0,\infty)$ and $L[f]$ is a function of $s$ defined for whatever values of $s$ lead to a convergent integral.

\example The Laplace Transform of $e^{t}$ is
\begin{align*}
L[e^t]& =\int_0^\infty e^t e^{-st} \ dt \\
& = \lim_{T \rightarrow \infty} \int_0^T e^{(1-s)t} \ dt \\
& = \lim_{T \rightarrow \infty} \left. \frac{e^{(1-s)t}}{1-s} \right|_0^T \\
& = \lim_{T \rightarrow \infty} \frac{1}{1-s} - \frac{e^{(1-s)T}}{1-s} \\
& = \frac{1}{1-s} \ \ \ \mbox{for} \ s >1.
\end{align*}
\qed

\exercise Calculate the Laplace Transform of the functions $t^2$, $\sin(t)$ and $e^{at}$ (where $a$ is a constant).

\exercise Prove that the Laplace Transform is linear: for any functions $f$ and $g$ and for any constant coefficients $a$ and $b$, $L[af+bg]=aL[f]+bL[g]$.  {\it (This only needs to hold on the set of $s$-values for which $L[f]$ and $L[g]$ are both defined.)}

\bigskip
It is typical to denote a transform of a function with a capital letter.  For example, when it is useful to display the variable, we will often denote the Laplace Transform of a function $f(t)$ by $F(s)$; otherwise we will write it as $L[f]$.  We usually do not care what the exact domain is for $F(s)$ -- it will be enough to know that there is some interval for $s$ on which the integral defining the transform converges.  The next theorem provides such a guarantee.

{\bf Theorem: }
Suppose there exists $M\geq 0$ and $a>0$ such that $f(t) \leq Me^{at}$ for all $t \geq 0$.  Then the integral defining the Laplace Transform converges for all $s >a$.

A function that satisfies the hypothesis of this theorem is said to be of {\bf exponential order}, because it does not grow any faster than exponential functions can grow.

\bigskip
\proof

Observe that for $s>a$ we have
\begin{align*}
\int_0^\infty | f(t) e^{-st} | \ dt & = \int_0^\infty | f(t)| e^{-st} \ dt \\
& \leq \int_0^\infty Me^{at} e^{-st} \ dt \\
& =  \int_0^\infty M e^{(a-s)t} \ dt \\
& = \frac{M}{s-a} \\
& < \infty.
\end{align*}
This proves that the integral defining $L[f]$ converges absolutely for all $s > a$.
\qed



\bigskip
Next, we introduce the key fact which allows us to use Laplace Transforms for solving initial value problems.  There is a close relationship between the Laplace transform of a function and that of its derivative:

\begin{center}
\fbox{
\begin{minipage}{1.5in}
\[ L[ f' ] = s L [f] - f(0) \]
\end{minipage}
}
\end{center}

We call this a reduction formula for the Laplace Transform because it allows us to ``reduce''  $L[y']$ to an expression involving $L[y]$.

\bigskip
\noindent
{\bf Theorem:} If $L[f]$ exists for $s>a$, and if $\lim_{t \rightarrow \infty} f(t)e^{-st}=0$ for $s > a$, then $L[f']$ also exists for $s>a$ and $L[f']=sL[f]-f(0)$.

\bigskip
Notice that any function of exponential order satisfies both hypotheses of this theorem.

\bigskip
\proof We use integration by parts, integrating $f'(t)$ and differentiating $e^{-st}$:
\begin{align*}
L[f'] & = \int_0^\infty f'(t) e^{-st} \ dt \\
& = \lim_{T \rightarrow \infty} \int_0^T f'(t) e^{-st} \ dt \\
& = \lim_{T \rightarrow \infty} \left[ e^{-st} f(t) - \int -se^{-st} f(t) \ dt \right]_0^T \\
& = \lim_{T \rightarrow \infty} e^{-sT}f(T) - e^{0t} f(0) + s \int_0^T f(t) e^{-st} \ dt \\
& = -f(0) + s \int_0^\infty f(t) e^{-st} \ dt \\
& = -f(0)+sL[f].
\end{align*} 
\qed 

In practice, when faced with an unknown function we will always assume that it is of exponential order and therefore satisfies hypotheses of these two theorems.  Of course, in theory such an assumption could lead to erroneous results, but in practical applications this rarely happens.  And because the process we illustrate in the next few examples furnishes us with a concrete function, we can always check it to make sure it satisfies the differential equation at hand.

To make use of the Laplace Transform to solve an initial value problem, we need to make use of one more fact which we will not prove:

\bigskip
\noindent
{\bf Theorem:} If $f$ and $g$ are continuous functions on $[0,\infty)$ and $L[f]=L[g]$, then $f=g$.

\bigskip
The point of this theorem is that the Laplace Transform is invertible.  We denote the inverse by $L^{-1}$.  Because $L$ is linear, so is $L^{-1}$: 
\[ L^{-1}[aF(s)+bG(s)] = aL^{-1} [F(s)] + b L^{-1}[G(s)]\]

\example Since $L[e^{2t}]=\frac{1}{s-2}$, it follows that $L^{-1} \left[ \frac{1}{s-2} \right] = e^{2t}.$
\qed

\exercise Find $L^{-1} \left[ \frac{1}{s^3} \right]$.  {\it (Hint: Refer to Exercise 1.)}

We now have enough machinery to use the Laplace Transform for solving an IVP.

\example Suppose $y$ is a solution of $y'+2y=0, \ y(0)=3$ on the domain $[0,\infty)$.  We take the Laplace Transform of both sides of the ODE:
\[ L[y'+2y] = L[0].\]
Then we use the facts that $L$ is linear and $L[0]=0$:
\[ L[y']+2L[y]=0.\]
Next we apply the formula for the Laplace Transform of a derivative:
\[ sL[y]-y(0)+2L[y]=0.\]
Insert the initial condition $y(0)=3$ and collect like terms:
\[ (s+2)L[y]-3=0.\]
Isolate $L[y]$:
\[ L[y] = \frac{3}{s+2}.\]
Finally, isolate $y$ by taking the inverse Laplace Transform of both sides:
\begin{align*}
y & = L^{-1} \left[ \frac{3}{s+2} \right] \\
& = 3 L^{-1} \left[ \frac{1}{s-(-2)} \right] \\
& = 3 e^{-2t}.
\end{align*}
This is the solution of the IVP above.
\qed

Clearly it will be useful to have a list of functions and their corresponding Laplace Transforms.  Here is a short list of such correspondences.

\begin{center}
\begin{tabular}{c|c}
$f(t)$ & $F(s)$ \\ \hline
$t^n$ & $\frac{n!}{s^{n+1}}$ \\
$e^{at}$ & $\frac{1}{s-a}$ \\
$\sin(kt)$ & $\frac{k}{s^2+k^2}$ \\
$\cos(kt)$ & $\frac{s}{s^2+k^2}$ \\
$\sinh(kt)$ & $\frac{a}{s^2-k^2}$ \\
$\cosh(kt)$ & $\frac{s}{s^2-k^2}$ \\
$e^{at}\sin(bt)$ & $\frac{b}{(s-a)^2+b^2}$ \\
$e^{at}\cos(bt)$ & $\frac{s-a}{(s-a)^2+b^2}$\\
$ t^n e^{at}$ & $\frac{n!}{(s-a)^{n+1}}$
\end{tabular}
\end{center}

\exercise Use Laplace Transforms to solve the IVP $y'+4y=6, \ y(0)=2$.    
 


Higher-order ODE can be solved in the same way.  When we transform $y''$, we just use the reduction formula twice:
\[ L[y'']=sL[y']-y'(0)=s(sL[y]-y(0))-y'(0)=s^2L[y]-sy(0)-y'(0).\]
The reader may choose to memorize this formula as well, or just to use the first-order formula repeatedly when required.

\example {\bf Solve the IVP $y''+9y=2, \ y(0)=1, \ y'(0)=0$.}

Transform both sides of the equation, rewrite all the Laplace Transforms in terms of $L[y]$, and then isolate $L[y]$:

\begin{align*}
L[y''+9y]=L[2] \\
L[y'']+9L[y]=\frac{2}{s} \\
sL[y']-y'(0)+9L[y]=\frac{2}{s} \\
s(sL[y]-y(0))-y'(0)+9L[y]=\frac{2}{s} \\
(s^2+9)L[y]-s-0=\frac{2}{s}\\
(s^2+9)L[y]=s+\frac{2}{s} \\
L[y]=\frac{s}{s^2+9}+\frac{2}{s(s^2+9)}
\end{align*}

Use a partial fractions decomposition to rewrite the right side of the equation:
\begin{align*} 
L[y]& =\frac{s}{s^2+9}+\frac{(2/9)}{s}+\frac{(-2/9)s}{s^2+9} \\
& = \frac{(7/9)s}{s^2+9}+\frac{(2/9)}{s}
\end{align*}
Then isolate $y$ using the inverse transform:
\begin{align*}
y & = L^{-1} \left[ \frac{(7/9)s}{s^2+9}+\frac{(2/9)}{s} \right] \\
& = \frac{7}{9} L^{-1} \left[ \frac{s}{s^2+9} \right] + \frac{2}{9} L^{-1} \left[ \frac{1}{s} \right] \\
& = \frac{7}{9} \cos(3t) +\frac{2}{9}
\end{align*}
\qed


\exercise Use Laplace Transforms to solve the IVP $y''+25y=t$, $y(0)=0$, $y'(0)=3$.

\exercise Use Laplace Transforms to solve the IVP $y''+4y'=6, \ y(0)=0, \ y'(0)=1$.    

\exercise Use Laplace Transforms to solve the IVP $y''-6y'+8y=6, \ y(0)=2, y'(0)=0$.   

\exercise Use Laplace Transforms to solve the IVP $y''+4y'+4y=\sin(t), \ y(0)=0, y'(0)=0$.



%% Cut below here for the book form.

\begin{center} {\LARGE Problems} \end{center}

\setcounter{problem}{1}


\problem Prove that, if $L[f(t)]=F(s)$, then $L[t^nf(t)]=(-1)^n F^{(n)}(s)$.  {\it (Hint: Use integration by parts.)}


\problem Prove all the formulas for the Laplace Transform given in the table in this chapter.

\problem Another useful transform in the study of differential equations is the Fourier Transform which can be defined for a function $f(t)$ by the formula $F[f] = \int_{-\infty}^\infty f(t) e^{-2\pi i \xi t} \ dt$.  (Here, the transform is a function of $\xi$.)  Verify the following reduction formula for differentiable functions $f$ that satisfy $\lim_{t \rightarrow \pm \infty}f(t)=0$:
\[ F[f'] = \frac{1}{2 \pi i} F[f].\]

\problem Prove that $f(t)=e^{(t^2)}$ {\it is not} of exponential order.

 






\end{document}