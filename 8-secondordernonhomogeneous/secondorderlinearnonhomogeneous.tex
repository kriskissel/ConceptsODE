\documentclass[12pt,letterpaper,twoside]{amsart}
\usepackage[latin1]{inputenc}
\usepackage{amsmath}
\usepackage{amsfonts}
\usepackage{graphicx}
\usepackage{amssymb}
\usepackage{multicol}
\usepackage{ulem}
\newcounter{example}
\newcounter{exercise}
\newcounter{problem}
\newcommand{\example}{\bigskip \noindent {\large {\sc Example \arabic{example}:}} \addtocounter{example}{1}}
\newcommand{\exercise}{\bigskip \noindent {\large {\sc Exercise \arabic{exercise}:}} \addtocounter{exercise}{1}}
\newcommand{\problem}{\bigskip \noindent {\large {\sc Problem \arabic{problem}:}} \addtocounter{problem}{1}}
\newcommand{\tech}{\marginpar{\vskip 10mm \begin{center}\includegraphics[width=0.25in]{calculatorimagesmall.eps} \end{center}}}
\newcommand{\solution}{\medskip \noindent {\bf Solution: }}
\newcommand{\R}{\mathbb{R}}





\begin{document}

\sffamily

%%%%  switch the commenting on this line and the next \chapter{Introduction}
\begin{center} {\LARGE The Method of Undetermined Coefficients} \end{center}

\setcounter{example}{1}
\setcounter{exercise}{1}


Now that we can solve second-order constant-coefficient linear homogeneous ODE of the form 
\[ a\ddot{y}+b\dot{y}+cy=0,\]
we would like to be able to solve the {\bf non-homogeneous} equations:
\[ a\ddot{y}+b\dot{y}+cy=f(t).\]
It is possible to write down general representation formulas for any continuous {\bf driving function} $f(t)$, but we will mostly be interested in the special cases when $f(t)$ is a polynomial, exponential or trigonometric function.

We will develop the idea of our technique in the following example.  Later examples will illustrate the streamlined version of this process.

\example Consider the differential equation 
\[y''+2y'+y=x^2.\]
We would like to find a general solution of this differential equation.  We will start by trying to find {\it one} solution.

What kinds of functions might satisfy the equation?  The driving function is a power function, and in order that a function, its derivative and second derivative might simplify on the left side of the ODE to just $x^2$, it would be a reasonable guess that some polynomial function might work.  We will therefore try to find a function of the form $y_p=Ax^2+Bx+C$ that satisfies the ODE.  (Notice that we don't want to try a polynomial of degree 3 or higher because there would be no way for the higher degree terms to cancel out and leave just $x^2$.)  Substitute this into the ODE to obtain
\begin{align*}
x^2 & = y_p''+2y_p'+y_p \\
& = (2A)+2(2Ax+B)+(Ax^2+Bx+C) \\
& = Ax^2 + (4A+B)x+(2A+2B+C)
\end{align*}
Equating the polynomial coefficients on both sides of the equation gives
\[ A=1, \ \ \ 4A+B=0, \ \ \ 2A+2B+C=0.\]
Consequently $A=1$, $B=-4$ and $C=6$.  Thus
\[ y_p (x) = x^2-4x+6\]
is one solution of the ODE.  Now let $y(x)$ be any solution of the equation, and define $y_h=y-y_p$.  Inserting this into the differential equation, we see that
\begin{align*}
x^2 & = y''+2y'+y \\
& = (y_h+y_p)''+2(y_h+y_p)'+(y_h+y_p) \\
& = y_h''+2y_h'+y_h + y_p''+2y_p'+y_p \\
& = y_h''+2y_h'+y_h + x^2
\end{align*}
Subtracting $x^2$ from both sides, we see that
\[ 0 = y_h''+2y_h'+y_h,\]
and we know how to solve this equation:
\[ y_h(x)=Ae^{-x}+Bxe^{-x}.\]
Consequently,
\[ y(x)=y_p(x)+y_h(x) =x^2-4x+6+Ae^{-x}+Bxe^{-x}.\]
All solutions of the ODE can be written in this form, so this is the general solution of the differential equation.
\qed

In the previous example, we too advantage of the following fact: if $y_p$ satisfies the non-homogeneous differential equation
\[ ay''+by'+cy=f(x)\]
and if $y_h$ is the general solution of the homogeneous equation
\[ ay''+by'+cy=0,\]
then $y=y_p+y_h$ is the general solution of the non-homogeneous ODE.  Based on this fact, we can try to find general solutions of non-homogeneous equations by finding just one solution (which we call a {\bf particular solution)} and then adding to it the general solution of the related homogeneous equation.

Our method for finding a particular solution was to guess a form of a particular solution (such as the polynomial $Ax^2+Bx+C$ we tried in the first example), and then by substituting it into the ODE we find the appropriate values for the unknown coefficients.  This approach is called the {\bf method of undetermined coefficients}.

\example {\bf Solve the IVP: $y''+y'-6y=3x+4$, $y(0)=1$, $y'(0)=0$.}

The homogeneous equation $y''+y'-6y=0$ has characteristic equation $r^2+r-6=0$, and the roots of this are $r=-3,2$.  Thus the homogeneous equation has the general solution $y_h=Ae^{-3x}+Be^{2x}$.  We guess that there might be a particular solution of the non-homogeneous equation of the form $y_p=Cx+D$.  Inserting this into the non-homogeneous equation yields
\[ 3x+4 = (0)+(C)-6(Cx+D) = -6Cx+(C-6D).\]
Equating coefficients gives us $C=-\frac{1}{2}$ and then $D=-\frac{7}{12}$.  This gives us $y_p=-\frac{1}{2}x-\frac{7}{12}$, and adding this to the general solution of the homogeneous equation yields the general solution of the non-homogeneous equation:
\[ y=-\frac{1}{2}x-\frac{7}{12}+Ae^{-3x}+Be^{2x}.\]
The initial conditions allow us to solve for $A$ and $B$:
\[ y(0)=1 \implies -\frac{7}{2} +A+B=1 \]
\[ y'(0)=0 \implies -\frac{1}{2}-3A+2B=0 \]
The solution of this system of algebraic equations is $A=\frac{17}{10}$ and $B=\frac{14}{5}$.  This gives us the solution of the IVP:
\[ y=-\frac{1}{2}x -\frac{7}{2}+\frac{17}{10}e^{-3x}+\frac{14}{5}e^{2x}.\]
\qed

\exercise Solve the initial value problem $y''-5y'+6y=x$, $y(0)=0$, $y'(0)=0$.

\exercise Try to find a particular solution to $y''+6y'=x$ of the form $y_p=Cx+D$.  End up proving that no such solution exists.

The last exercise shows us how we might need to be more clever when guessing the form of our particular solution.  If Any term in the driving function is a solution of the related homogeneous equation, we will need to modifythe form of our guess.  For the differential equation in the last exercise, the correct form of the guess is actually a degree-two polynomial.

\example Consider the ODE $y''+6y'=x$.  Let us seek a solution of the form $y_p=Cx^2+Dx$.  Insertin g this into the differential equation produces
\[ x=(2C)+6(2Cx+D)=12Cx+(2C+D).\]
Equating coefficients gives us $C=\frac{1}{12}$ and then $D=-\frac{1}{6}$.  Now we see that the function $y_p=\frac{1}{12}x^2-\frac{1}{6}x$ is a solution.
\qed

The general principle we follow here is this: if the driving term of the non-homogeneous equation is a polynomial (or a monomial) of degree $N$, then our guess for the form of a particular solution is 
\[ y_p=x^S q(x),\]
where $q(x)$ is a polynomial of degree $N$, and where $S \geq 0$ is the smallest non-negative integer such that no term in the polynomial $x^S q(x)$ is a solution of the related homogeneous equation.

Because we need to compare our guess with solutions of the related homogeneous equation, it is a good practice to find the general solution of the homogeneous equation first.

\exercise Find the general solution of $y''+2y'=x^2$.

We can employ the same techniques when the driving term is an exponential function.

\example {\bf Find the general solution of $y''+2y'+y=e^{2x}$.}

The characteristic equation is $r^2+2r+1=0$, which has a repeated root $r=-1$.  Thus the general solution of the related homogeneous equation is $y_h=Ae^{-x}+Bxe^{-x}$.  Next we guess that a particular solution of the non-homogeneous equation will have the form $y_p=Ce^{2x}$:
\[ e^{2x}=(4e^{2x})+2(2e^{2x})+(e^{2x}) = 9e^{2x},\]
so that $C=\frac{1}{9}$.  Therefore the general solution of the non-homogeneous equation is
\[ y=\frac{1}{9}e^{2x}+Ae^{-x}+Bxe^{-x}.\]
\qed

\example {\bf Find the general solution of $y''+2y'+y=e^{-x}$.}

The general solution of the related homogeneous equation is $y_h=Ae^{-x}+Bxe^{-x}$.  Therefore no multiple of $e^{-x}$ can be a solution of the non-homogeneous equations.  Also, nether can any multiple of $xe^{-x}$.  However, we can find a solution by looking for a multiple of $x^2e^{-x}$.  Let $y_p=Cx^2e^{-x}$.  Then $y_p'=C(2x-x^2)e^{-x}$ and $y_p''=C(2-4x+x^2)e^{-x}$.  Insert these into the ODE:
\begin{align*} e^{-x}& =\left(C(2-4x+x^2)e^{-x}\right)+2\left(C(2x-x^2)e^{-x}\right) +\left(Cx^2e^{-x}\right) \\
& = 2Ce^{-x}
\end{align*}
Thus $C=\frac{1}{2}$.  So $y_p=\frac{1}{2}x^2e^{-x}$ is a particular solution, and therefore the general solution is
\[ y=\frac{1}{2}x^2 e^{-x} +Ae^{-x}+Bxe^{-x}.\]
\qed

As in Example 3, when we recognized that the natural guess would be a solution of the homogeneous equation, we modified it by multiplying by the smallest power of $x$ such that the product would not be a homogeneous solution.  This same approach can applied when the driving terms is a sine or cosine function.  In general, if the driving term is $\sin(mx)$ or $\cos(mx)$, our guess will be a function of the form $y_p=Asin(mx)+B\cos(mx)$, unless we need to multiply by a power of $x$ to ensure that no term in our guess is a homogeneous solution.

\example {\bf Find a general solution of $y''-y=\sin(2x)$.}

The characteristic equation is $r^2-1=0$, which has solutions $r=\pm 1$; thus the solution of the homogeneous equation is $y_h=Ae^x+Be^{-x}$.  Now we guess that a solution of the non-homogeneous equation might have the form $y_p=C\sin(2x)+D\cos(2x)$.  Inserting this into the ODE yields
\begin{align*}
\sin(2x) & = \left(-4C\sin(2x)-4D\cos(2x)\right)-\left(C\sin(2x)+D\cos(2x) \right) \\
& = -5C\sin(2x)-5D\cos(2x).
\end{align*}
Equating coefficients gives us $C=-\frac{1}{5}$ and  $D=0$, so $y_p=-\frac{1}{5}\sin(2x)$.  The general solution is thus
\[ y=-\frac{1}{5}\sin(2x)+Ae^x+Be^{-x}.\]
\qed

\example {\bf Find a general solution of $y''+y=\sin(2x)$.}

The characteristic equation is $r^2+1=0$, which has solutions $r=\pm i$.  We thus write the general solution of the homogeneous equation as $y_h=A\sin(x)+B\cos(x)$.  Suppose a particular solution is $y_p=C\sin(2x)+D\cos(2x)$.  Then
\begin{align*}
\sin(2x)&=\left(-4A\sin(2x)-4B\cos(2x) \right) + \left(A\sin(2x)+\cos(2x) \right) \\
& = -3C\sin(2x)-3D\cos(2x).
\end{align*}
So $C=-\frac{1}{3}$ and $D=0$.  Thus $y_p=-\frac{1}{3}\sin(2x)$ and
\[ y = -\frac{1}{3}\sin(2x)+A\sin(x)+B\cos(x).\]
\qed

\example {\bf Find a general solution of $y''+y=\sin(x)$.}

As in the previous example, $y_h=A\sin(x)+B\cos(x)$.  Now, because the driving function $\sin(x)$ is a solution of the homogeneous equation, we use the guess $y_p=Cx\sin(x)+Dx\cos(x)$:
\begin{align*}
\sin(x) & = \left(2C \cos(x)-Cx\sin(x)-2D\sin(x)-D\sin(x)\right) \\
& \ \ \ \ \ \ \ \ \ \ +\left(Cx\sin(x)+Dx\cos(x) \right) \\
& = 2C\cos(x)-2D\sin(x).
\end{align*}
Therefore $C=0$ and $D=-\frac{1}{2}$, $y_p=-\frac{1}{2}x\cos(x)$ and 
\[ y = -\frac{1}{2}x\cos(x)+A\sin(x)+B\cos(x).\]
\qed

\exercise Find one solution for each of the following differential equations.
\begin{itemize}
\item $y''-y'+4y=x^2+1$
\item $y''+2y'+3y=\sin(3t)$
\item $y''+6y'=\cos(t)$
\item $\ddot{w}-\dot{w}-3w=e^{t}$
\item $v''+v=\sin(t)$
\item $y''-5y'+6y=e^{2x}$
\end{itemize}

\exercise Solve the following initial value problems.
\begin{itemize}
\item $y''+y=e^{3t}, \ y(0)=0, \ y'(0)=0$
\item $y''+6y'+9y=\sin(2t), \ y(0)=1, \ y'(0)=0$
\item $y''-4y=e^{-2t}, \ y(0)=0, y'(0)=0$
\end{itemize}

%% Cut below here for the book form.

\begin{center} {\LARGE Problems} \end{center}

\setcounter{problem}{1}

\problem Solve the IVP $y''+y=f(t), \ y(0)=0, y'(0)=2$, where $f$ is the function

\[ f(t) = \left\{ \begin{matrix} 0 & \mbox{for} \ t \leq \pi \\ t-\pi & \mbox{for} \ t > \pi \end{matrix} \right. .\]

{\it (Hint: Split this into two separate initial value problems -- one for $t \leq \pi$ and one for $t \geq \pi$.  You'll need to solve the first problme in order to know what initial conditions to use for the second problem.)}






\end{document}