\documentclass[12pt,letterpaper,twoside]{amsart}
\usepackage[latin1]{inputenc}
\usepackage{amsmath}
\usepackage{amsfonts}
\usepackage{graphicx}
\usepackage{amssymb}
\usepackage{multicol}
\usepackage{ulem}
\newcounter{example}
\newcounter{exercise}
\newcounter{problem}
\newcommand{\example}{\bigskip \noindent {\large {\sc Example \arabic{example}:}} \addtocounter{example}{1}}
\newcommand{\exercise}{\bigskip \noindent {\large {\sc Exercise \arabic{exercise}:}} \addtocounter{exercise}{1}}
\newcommand{\problem}{\bigskip \noindent {\large {\sc Problem \arabic{problem}:}} \addtocounter{problem}{1}}
\newcommand{\tech}{\marginpar{\vskip 10mm \begin{center}\includegraphics[width=0.25in]{calculatorimagesmall.eps} \end{center}}}
\newcommand{\solution}{\medskip \noindent {\bf Solution: }}
\newcommand{\R}{\mathbb{R}}





\begin{document}

\sffamily

%%%%  switch the commenting on this line and the next \chapter{Introduction}
\begin{center} {\LARGE The Method of Integrating Factors} \end{center}

\setcounter{example}{1}
\setcounter{exercise}{1}

We say an ordinary differential equation of order 1 is linear if it can be written in the form:
\begin{equation} a(t)y'(t)+b(t)y(t)=f(t) \label{firstorderlinear} \end{equation}
In this chapter, we will explore a technique for analytically solving these differential equations and other equations that can be written in this form.

\example Solve the initial value problem $\frac{dy}{dx} = 3-4y, \ y(0)=1$.

First, let us rewrite the differential equation in the form:
\[ \frac{dy}{dx}+4y=3.\]
Next, mutiply both sides of the equation by $e^{4x}$ to obtain
\[ e^{4x} \frac{dy}{dx} + 4e^{4x} y = 3e^{4x}.\]
The point of this last step is that the left side of the equation is now the derivative of $e^{4x}y$:
\[ \frac{d}{dx}\left[ e^{4x} y \right] = 3e^{4x}.\]
If we anti-differentiate both sides with respect to $x$, we obtain
\[ e^{4x} y = \int 3e^{4x} \ dx = \frac{3}{4} e^{4x} + C.\]
Isolating $y$ gives us
\[ y = \frac{3}{4} + Ce^{-4x}.\]
The initial condition $y(0)=1$, implies that $C = \frac{1}{4}$.  Therefore the solution of the IVP is
\[ y=\frac{3}{4}+\frac{1}{4}e^{-4x}.\]
\qed

Multiplying by the expression $e^{4x}$ is what allowed us to recognize the left side of the equation as the derivative of an expression (which would have come from using the product rule), and that was what allowed us to simplify when we integrated both sides of the equation.  For that reason, the expression is referred to as an {\bf integrating factor}.

Any first-order linear differential equation written in standard form, 
\begin{equation} \frac{dy}{dx}+p(x)y=q(x),\label{standardfirstorderlinear} \end{equation}
is a candidate for this {\bf method of integrating factors}.  Once an equation is written in this form, we multiply both sides by the integrating factor $e^{\int p(x) dx}$:
\[ e^{\int p(x) dx} \frac{dy}{dx} + p(x) e^{\int p(x) dx} y = q(x) e^{\int p(x) dx}.\]
Now we can use the product rule to recognize the left side as a derivative:
\[ \frac{d}{dx} \left[ e^{\int p(x) dx} y \right] = q(x) e^{\int p(x) dx}.\]
Anti-differentiate both sides to get
\[ e^{\int p(x) dx} y = \int q(x) e^{\int p(x) dx} \ dx,\]
and then isolate y:
\[ y = e^{-\int p(x) dx } \int q(x) e^{\int p(x) dx} \ dx.\]

The reader should not try to memorize this formula.  Instead, think of this as a general process that can be applied to solve the differential equation: 
\begin{enumerate} 
\item Write the first-order linear equation in standard form;
\item Multiply by an appropriate integrating factor of the form $e^{\int p(x) dx}$;
\item Use the product rule to rewrite the left side as a derivative;
\item anti-differentiate both sides;
\item isolate $y$.
\end{enumerate}

Note that, in practice, any anti-derivative of $p(x)$ will suffice when you construct an integrating factor, so we may ignore the constant of integration when we find $e^{\int p(x) dx}$.

\exercise Solve the initial value problem $y' = y+e^x, \ y(0)=3$.

\exercise Solve the initial value problem $\dot{y} = xy+x, \ y(0)=1$.


\example Consider a 100-gallon tank that begins full of pure water.  Salt water solution containing 50 grams of salt per gallon is added to the tank at a rate of 2 gallons per minute.  Simultaneously, the solution in the tank is kept thoroughly mixed, and the tank drains at a rate of 3 gallons per minute.  Find the quantity of salt in the tank after 10 minutes.

Let us begin by writing a differential equation that describes the rate at which the quantity of salt in the tank is changing over time.  Let $S(t)$ represent the number of grams of salt in the tank after $t$ minutes.  Then
\[ \frac{dS}{dt} = \mbox{(rate in) - (rate out)},\]
where {\it rate in} refers to the rate at which salt is being added to the tank and {\it rate out} refers to the rate at which salt is leaving the tank.  Salt enters at a rate of 
\[\mbox{rate in} = \left( \frac{50 \ grams}{1 \ gallon}\right) \left( \frac{2 \ gallons}{1 \ minute}\right) = \frac{100 grams}{minute}.\]
The concentration of salt in the tank at any given instant is $\frac{S(t)}{V(t)}$, where $V(t)$ is the volume of liquid in the tank after $t$ minutes.  Because the liquid enters the tank at 2 gallons per minute but leaves at 3 gallons per minute, the volume is decreasing at 1 gallon per minute.  Therefore the volume will be $V(t) = 100-t$ gallons, so the concentration of salt in the tank is $\frac{S}{100-t} \frac{grams}{gallon}$.  Consequently, the rate at which salt is leaving the tank is described by:
\[ \mbox{rate out} = \left( \frac{S}{100-t} \frac{grams}{gallon}\right)  \left(\frac{2 gallons}{1 minute}\right) = \frac{2S}{100-t} \frac{grams}{minute}.\]
Hence
\[ \frac{dS}{dt} = 100 - \frac{2S}{100-t} \ \frac{grams}{minute}.\]
Also, the tank initially contains pure water, so $S(0)=0$.  Now we will solve this initial value problem for $S$.

Write the equation in the form
\begin{equation} \frac{dS}{dt} + \frac{2}{100-t} S = 100. \label{example2} \end{equation}
The integrating factor we need here is
\begin{align*} e^{\int \frac{2}{100-t} dt} 
& = e^{-2 \ln |100-t|} \\
& = e^{\ln (100-t)^{-2}} \\
& = \frac{1}{(100-t)^2}.
\end{align*}
Multiply both sides of the differential equation by this factor:
\[ \frac{1}{(100-t)^2}\frac{dS}{dt} + \frac{2}{(100-t)^3} S = \frac{100}{(100-t)^2}.\]
Reversing the product rule, we write this as
\[ \frac{d}{dt} \left[ \frac{1}{(100-t)^2} S \right] = \frac{100}{(100-t)^2},\]
and anti-differentiation leads us to
\begin{align*} \frac{1}{(100-t)^2} S 
& = \int \frac{100}{(100-t)^2} \ dt \\
& = \frac{100}{100-t}+C.
\end{align*}
Isolating $S$ gives us
\[ S = 100(100-t)+C(100-t)^2.\]
The initial condition $S(0)=0$ implies $C=-1$, so the solution of the IVP is
\[ S= 100(100-t)-(100-t)^2 \ \mbox{grams}.\]
Finally,  we calculate $S(10)$:
\[ S(10) = 100(90)-(90)^2=900.\]
This tells us that there will be 900 grams of salt in the tank after 10 minutes.
\qed

It is important to note that the solution in the above example only makes physical sense for $0 \leq t \leq 100$ (after that the model we constructed would predict a negative volume of liquid in the tank).

The next example will illustrate how we can sometimes solve a non-linear differential equation by converting it into a related linear equation.

\example Find a solution of the initial value problem $\dot{y}=\frac{y}{t}+y^2, \ y(1)=\frac{1}{2}$ defined for positive numbers $t$.

This differential equation is not separable, and it is not linear.  However, we can find a related linear differential equation in the following way: let $u = \frac{1}{y}$.  Then we have
\begin{align*}
\dot{u} 
& = -\frac{1}{y^2} \dot{y} \ \ \ \mbox{(by the chain rule)} \\
& = -\frac{1}{y^2} \left( \frac{y}{t}+y^2 \right) \ \ \ \mbox{(by the differential equation $y$ must satisfy)} \\
& = -\frac{1}{ty} -1 \\
& = -\frac{u}{t}-1 \ \ \ \mbox{(since $u=y^{-1}$)}.
\end{align*}
Now we have a differential equation that $u$ must satisfy: $\dot{u}=-\frac{u}{t}-1$.  If we can solve this differential equation to find $u$, then we can take the reciprocal of that solution to find a formula for $y$. Rewrite this as
\[ \dot{u} + \frac{1}{t} u = -1.\]
Multiply both sides by the integrating factor $e^{\int \frac{1}{t} dt} = e^{\ln |t|}=|t|=t$ (since we are only asked to consider positive values for $t$):
\[ t\dot{u} + u = -t.\]
Reversing the product rule on the left side gives us
\[ \frac{d}{dt} \left[ t u \right] = -t.\]
Integrate both sides with respect to $t$:
\[ t u = \int -t \ dt = -\frac{t^2}{2} + C.\]
Isolate $u$:
\[ u = -\frac{t}{2}+\frac{C}{t}\]
Because $u(1)=\frac{1}{y(1)}=\frac{1}{1/2}=2$, we obtain $C=\frac{5}{2}$.  This gives us the formula $u = -\frac{t}{2}+\frac{5}{2t}$, and taking the reciprocal yields
\[ y = \frac{1}{-\frac{t}{2} + \frac{5}{2t}},\]
or
\[ y = \frac{2t}{5-t^2}.\]
Observe that the interval of definition for this solution is $(-\sqrt{5},\infty)$.
\qed

The process above can be modified for any differential equation of the form
\[ \frac{dy}{dx}=p(x)y+y^N,\]
where $N$ is a positive integer.  These are called {\bf Bernoulli equations.}  For any such equation, the substitution $u=y^{1-N}$ leads to the differential equation
\[ \frac{du}{dx} = -Np(x)u-N,\]
which is a candidate for the method of integrating factors.  Again, the reader should not think of this as a formula to memorize but as a general procedure for Bernoulli equations:
\begin{enumerate}
\item Let $u=y^{1-N}$; use the chain rule and the differential equation for $y$ to find a differential equation for $u$;
\item solve for $u$ (be sure to modify the initial condition for $y$ appropriately);
\item use the solution for $u$ to obtain a formula for $y$.
\end{enumerate}

\exercise Solve the Bernoulli equation $y'=y+y^5$ subject to the initial condition $y(1)=3$.






%% Cut below here for the book form.

\begin{center} {\LARGE Problems} \end{center}

\setcounter{problem}{1}

\problem Find the general solution of the differential equation $a\dot{y}+by=c$, for any constant coefficients $a, \ b, \ c$, with $a \neq 0$.  {\it (Hint: You should consider the cases $b=0$ and $b \neq 0$ separately.)}

\problem A large tank begins with 50 gallons of water into which is dissolved 10 grams of salt.  Salt water solution with a concentration of 5 grams of salt per gallon is added to the tank at a rate of 4 gallons per minute.  Meanwhile, the solution in the tank is thoroughly mixed and drains at a rate of 2 gallons per minute.  How long will it take until there are 1000 grams of salt in the tank?  How much liquid will be in the tank at that instant?


\problem A large object is dropped from an airplane.  The velocity $v$, measured in meters per second, satisfies
\[ \dot{v} = 9.8 -Kv,\]
where $K>0$ is a constant.  If the object is falling at $100 \frac{m}{s}$ after 10 seconds, determine how fast it will be falling after 20 seconds.

\problem A large object is dropped from an airplane at an altitude of 3000 meters.  The velocity $v$, measured in meters per second, satisfies
\[ \dot{v} = 9.8 -Kv,\]
where $K>0$ is a constant.  If the object falls 200 meters in the first 10 seconds, estimate when the object will hit the ground.  {\it (You will encounter an algebraic equation that cannot be solved analytically.  Solve it numerically or graphically, but keep as many decimal places of accuracy as you can throughout the problem-solving process.)}



\problem Solve the logistic differential equation $\dot{P}=kP(K-P)$ by treating it as a Bernoulli equation and making a substitution.

\problem The idea of substitution has application beyond Bernoulli equations.  For example, any differential equation of the form $y'=f(ax+by+c)$ can be transformed into a separable differential equation by means of the substitution $u=ax+by+c$.  Use this idea to solve the initial value problem:
\[ y'=(x+2y)^2, \ \ \ y(0)=1.\]

\problem Solve the initial value problem $y'=\sin^2(x-y), \ y(0)=1$.





\end{document}